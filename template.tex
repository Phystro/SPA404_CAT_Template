\documentclass[12pt, a4paper]{article}


%%
% These are the packages to be used in the code
%%
\usepackage[margin=2cm]{geometry}
\usepackage{braket}
\usepackage{amsmath, amsfonts, amsthm}
\usepackage{amssymb}
\usepackage{float}
\usepackage{graphicx}
\usepackage{authblk}
\usepackage{booktabs}
\usepackage{pgfplots}
\usepackage{fancyhdr}
\usepackage{array}
\usepackage{tikz}
\usepackage{verbatim}



%%%%%%%%%%%%%%%%%%%%%%%%%%%%%%%%%%%%%%%%%%%%%%
% Uncomment the \date{} and put your own custom date 
% Comment the \date{} to use the default date
%\date{}

%%%%%%%%%%%%%%%%%%%%%%%%%%%%%%%%%%%%%%%%%%%%%%
% Add your name and title and Course/Class name
% Kuna format tatu hapo, comment out yenye hutaki, uncomment out yenye unataka kutumia
% Badilisha and ueke mbwembwe zako to each format ili zisifanane
%
% FORMAT 1
%\author{Firstname Secondname Thirdname | I44/1234/2018}
%\title{\small SPA 404: Fundamentals of Space Flight\\
%	\large CAT}
%
% FORMAT 2
%\author{Firstname Secondname Thirdname \\ \textbf{I44/1234/2018}}
%\title{\small SPA 404: Fundamentals of Space Flight\\
%	\large CAT}
%
% FORMAT 3
\author{Firstname Secondname Thirdname}
\title{SPA 404 Fundamentals of Space Flight CAT \\ I44/1234/2018}
%
%%%%%%%%%%%%%%%%%%%%%%%%%%%%%%%%%%%%%%%%%%%%%%%

\begin{document}
	\maketitle
	
	% Specify your satellite name hapa
	\textbf{Satellite: ANDIKA JINA LA SATELAITI YAKO HAPA}
	
	\section*{Question 1}
	
	\begin{enumerate}
		% i
		\item Make a table (using Lyx or Texstudio) of the Two-Line Element Set for your satellite
		%%%%%%%%%%%%%%%%%%%%%%%%%%%%%%%%%%%%%%%%%%%%%%%%%%%%%%%%
		%%% THE TABLE BEGINS HERE
		\begin{table}[ht]
			\centering
			\caption{TWO LINE MEAN ELEMENT SET - SATELLITE NAME EKA PIA HAPA}
			{\renewcommand{\arraystretch}{2}
				\begin{tabular}[t]{|c|c|c|c|c|c|c|c|c|}
					%%%%%%%%%%%%%%%%%%%%%%%%%%%%%%
					%% EKA VALUES ZA TAL YAKO KWA HIZO SPACES IN BETWEEN THE & SYMBOL
					\hline
					1	&	&	&	&	&	&	&	&	\\	% First rrow
					\hline
					2	&	&	&	&	&	&	&	&	\\	% Second row
					\hline
			\end{tabular}} \quad
			\label{tab:caption}
		\end{table}
	%%% THE TABLE ENDS HERE
		
		
		% ii
		\item Make another table for six Keplerian elements table
		\begin{table}[ht]
			\centering
			\caption{Six Keplerian Elements}
			{\renewcommand{\arraystretch}{2}
				\begin{tabular}[t]{|c|c|}
					%%%%%%%%%%%%%%%%%%%%%%%%%%%%%%%%%%%%
					%% MJIJAZIE VALUES HAPO KWA COLUMN YA VALUE, IN BETWEEN THE & SYMBOL NA THE \\ SYMBOL
					\hline
					\textbf{Keplerian Element}  & \textbf{Value}\\
					\hline
					Semi major axis  & 0 km\\
					\hline
					Eccentricity & 0.0\\
					\hline
					True anomaly & 0.0$^\circ$\\
					\hline
					Inclination & 0.0$^\circ$\\
					\hline
					Right Ascension (longitude of the ascending node)  & 0.0$^\circ$\\
					\hline
					Argument of perigee  & 0.0$^\circ$\\
					\hline
			\end{tabular}} \quad
			\label{tab:caption2}
		\end{table}
		
		
		
		\pagebreak
		% III
		%%%%%%%%%%%%%%5
		%%% KAMA KAWAIDA ONGEZA MBWE MBWE ZAKO HAPA ZISIFANANE
		\item Derive and edit the equation of an ellipse
		\\
		\text{Let the ellipse be centered at the origin $(0, 0)$}\\
		\text{Let the foci be on the x-axis at $(-c, 0)$ and $(c, 0)$.}\\
		\text{The vertices are also on the x-axis at $(-a, 0)$ and $(a, 0)$.}\\
		An ellipse is the set of all points $(x, y)$ such that the sum of all distances from $(x, y)$ to te foci is constant.\\
		Let's define:
		\begin{align*}
			d_1 &= \text{distance from} \; (-c, 0) \; \text{to} \; (x, y)\\
			d_2 &= \text{distance from} \; (c, 0) \; \text{to} \; (x, y)
		\end{align*}
		We can allow $(x, y)$ to be on the x-axis to help us get the constant. Thus, $d_1$ is the distance between $(-c, 0)$ and $(a, 0)$ while $d_2$ is distance between $(c, 0)$ and $(a, 0)$.
		\begin{align*}
			d_1 &= \sqrt{(a - -c)^2 + (0 - 0)^2} = \sqrt{(a + c)^2} = a + c\\
			d_2 &= \sqrt{(a - c)^2 + (0-0)^2} = \sqrt{(a - c)^2} = a - c\\
			\therefore d_1 + d_2 &= 2a
		\end{align*}
		Finding expressions for $d_1$ and $d_2$. Using the distance formula:
		\begin{align*}
			d_1 + d_2 &= \Big( \sqrt{(x - -c)^2 + (y - 0)^2} \Big) + \Big( \sqrt{(x - c)^2 + (y-0)^2} \Big) = 2a
		\end{align*}
		\begin{align*}
			2a &= \Big( \sqrt{(x +c)^2 + y^2} \Big) + \Big( \sqrt{(x - c)^2 + y^2} \Big) \\
			\sqrt{(x +c)^2 + y^2} &= 2a - \sqrt{(x - c)^2 + y^2}\\
			(x + c)^2 + y^2 &= \Big[2a -  \sqrt{(x - c)^2 + y^2} \Big]^2\\
			\text{expanding the squares}\\
			x^2 + c^2 + 2xc + y^2 &= 4a^2 - 4a\sqrt{(x - c)^2 + y^2}  + y^2 + x^2 + c^2 - 2xc\\
			4xc &= 4a^2 - 4a\sqrt{(x - c)^2 + y^2}\\
			4(xc - a^2) &= 4a\sqrt{(x - c)^2 + y^2}\\
			xc - a^2 &= a\sqrt{(x - c)^2 + y^2}\\
			\Big[xc - a^2\Big]^2 &= \Big[a\sqrt{(x - c)^2 + y^2}\Big]^2\\
			x^2c^2 + a^4 - 2xca^2 &= a^2(y^2 + x^2 + c^2 - 2xc)\\
			x^2c^2 + a^4 - 2xca^2 &= a^2y^2 + a^2x^2 + a^2c^2 - 2xca^2\\
			x^2c^2 - a^2x^2 &= a^2y^2 + a^2c^2 - a^4\\
			x^2(c^2 - a^2) &= a^2y^2 + a^2(c^2 - a^2)\\
			x^2(c^2 - a^2)  - a^2(c^2 - a^2) &= a^2y^2 \\
			(x^2 - a^2)(c^2 - a^2) &=  a^2y^2\\
			\text{Set a new variable: $b^2 = a^2 - c^2$}\\
			-(x^2 - a^2)b^2 &= a^2y^2\\
			-x^2b^2 + a^2b^2 &= a^2y^2\\
			x^2b^2 + a^2y^2 &= a^2b^2\\
			\frac{x^2}{a^2} + \frac{y^2}{b^2} &= 1
		\end{align*}
		Therefore, equation of an ellipse centered around the origin is:
		\begin{align}
			\frac{x^2}{a^2} + \frac{y^2}{b^2} &= 1
		\end{align}
	
	
%%%%%%%%%%%%%%%%%%%%%%%%%%%%%%%%%%%%%%%%%%%%%%%%%%%%%
%%%%%%%%%%%%%%%%%%%%%%%%%%%%%%%%%%%%%%%%%%%%%
%%%%%% PIA UNAWEZA UNCOMMENT HII HAPA UITUMIE FOR EQUATION FOAN ELLIPSE BADALA YA HIYO YENYE IKO HAPO JUU
%%%%%% JUST AS AN ALTERNATIVE - YOUR CHOICE
%%%%%%%%%%%%%%%%%%%%%%%%%%%%%%%%%%%%%%%%%%%%%%%%%%%%%%%%%%%%%%%%%%%%%%
%	\text{Equation of an ellipse}\\\\
%	\begin{array}
%		{ll}\text{ }{d}{1}+{d}{2}=\sqrt{{\left(x-\left(-c\right)\right)}^{2}+{\left(y - 0\right)}^{2}}+\sqrt{{\left(x-c\right)}^{2}+{\left(y - 0\right)}^{2}}=2a\hfill & \text{Distance formula}\hfill \\ \sqrt{{\left(x+c\right)}^{2}+{y}^{2}}+\sqrt{{\left(x-c\right)}^{2}+{y}^{2}}=2a\hfill & \text{Simplify expressions}\text{.}\hfill \\ \text{ }\sqrt{{\left(x+c\right)}^{2}+{y}^{2}}=2a-\sqrt{{\left(x-c\right)}^{2}+{y}^{2}}\hfill & \text{Move radical to opposite side}\text{.}\hfill \\ \text{ }{\left(x+c\right)}^{2}+{y}^{2}={\left[2a-\sqrt{{\left(x-c\right)}^{2}+{y}^{2}}\right]}^{2}\hfill & \text{Square both sides}\text{.}\hfill \\ \text{ }{x}^{2}+2cx+{c}^{2}+{y}^{2}=4{a}^{2}-4a\sqrt{{\left(x-c\right)}^{2}+{y}^{2}}+{\left(x-c\right)}^{2}+{y}^{2}\hfill & \text{Expand the squares}\text{.}\hfill \\ \text{ }{x}^{2}+2cx+{c}^{2}+{y}^{2}=4{a}^{2}-4a\sqrt{{\left(x-c\right)}^{2}+{y}^{2}}+{x}^{2}-2cx+{c}^{2}+{y}^{2}\hfill & \text{Expand remaining squares}\text{.}\hfill \\ \text{ }2cx=4{a}^{2}-4a\sqrt{{\left(x-c\right)}^{2}+{y}^{2}}-2cx\hfill & \text{Combine like terms}\text{.}\hfill \\ \text{ }4cx - 4{a}^{2}=-4a\sqrt{{\left(x-c\right)}^{2}+{y}^{2}}\hfill & \text{Isolate the radical}\text{.}\hfill \\ \text{ }cx-{a}^{2}=-a\sqrt{{\left(x-c\right)}^{2}+{y}^{2}}\hfill & \text{Divide by 4}\text{.}\hfill \\ \text{ }{\left[cx-{a}^{2}\right]}^{2}={a}^{2}{\left[\sqrt{{\left(x-c\right)}^{2}+{y}^{2}}\right]}^{2}\hfill & \text{Square both sides}\text{.}\hfill \\ \text{ }{c}^{2}{x}^{2}-2{a}^{2}cx+{a}^{4}={a}^{2}\left({x}^{2}-2cx+{c}^{2}+{y}^{2}\right)\hfill & \text{Expand the squares}\text{.}\hfill \\ \text{ }{c}^{2}{x}^{2}-2{a}^{2}cx+{a}^{4}={a}^{2}{x}^{2}-2{a}^{2}cx+{a}^{2}{c}^{2}+{a}^{2}{y}^{2}\hfill & \text{Distribute }{a}^{2}.\hfill \\ \text{ }{a}^{2}{x}^{2}-{c}^{2}{x}^{2}+{a}^{2}{y}^{2}={a}^{4}-{a}^{2}{c}^{2}\hfill & \text{Rewrite}\text{.}\hfill \\ \text{ }{x}^{2}\left({a}^{2}-{c}^{2}\right)+{a}^{2}{y}^{2}={a}^{2}\left({a}^{2}-{c}^{2}\right)\hfill & \text{Factor common terms}\text{.}\hfill \\ \text{ }{x}^{2}{b}^{2}+{a}^{2}{y}^{2}={a}^{2}{b}^{2}\hfill & \text{Set }{b}^{2}={a}^{2}-{c}^{2}.\hfill \\ \text{ }\frac{{x}^{2}{b}^{2}}{{a}^{2}{b}^{2}}+\frac{{a}^{2}{y}^{2}}{{a}^{2}{b}^{2}}=\frac{{a}^{2}{b}^{2}}{{a}^{2}{b}^{2}}\hfill & \text{Divide both sides by }{a}^{2}{b}^{2}.\hfill \\ \text{ }\frac{{x}^{2}}{{a}^{2}}+\frac{{y}^{2}}{{b}^{2}}=1\hfill & \text{Simplify}\text{.}\hfill
%	\end{array}
%%%%%%%%%%%%%%%%%%%%%%%%%%%%%%%%%%%%%%%%%%%%%%%%%%%%%%%%%%%%%%%%%%%%%%
%%%%%%%%%%%%%%%%%%%%%%%%%%%%%%%%%%%%%%%%%%%%%%%%%%%%%%%%%%%%%%%%%%%%%%
		
		% iv
		\pagebreak
		\item Derive the orbital equation for your satellite
		\text{Using the equation of motion for the two-body problem}
		\begin{align*}
			\mathbf{\ddot{r}} + \frac{\mu}{r^3} \mathbf{\vec{r}} &= 0
		\end{align*}
		Where:
		\begin{align*}
			\mu & \equiv GM\\
			\mathbf{\vec{r}} &= \text{seperation displacement between the Earth and the satellite}
		\end{align*}
		Cross product with the angular momentum, $\mathbf{\vec{L}}$
		\begin{align*}
			\mathbf{\ddot{r}} \times \mathbf{\vec{L}}&= \frac{\mu}{r^3}\\
			\text{Setting the equation as:}\\
			\frac{\mu}{r^3}(\mathbf{\vec{L}} \times \mathbf{\vec{r}}) &= \frac{\mu}{r^3} \equiv \frac{\mu}{r}\mathbf{\vec{v}} - \frac{\mu}{r^2}\dot{r}\mathbf{\vec{r}}
		\end{align*}
		Note that the derivative of the unit vector is:
		\begin{align*}
			\mu\frac{d}{dt}\Big(\frac{\mathbf{\vec{r}}}{r}\Big) &= \frac{\mu}{r}\mathbf{\vec{v}} - \frac{\mu}{r^2}\dot{r}\mathbf{\vec{r}}
		\end{align*}
		Allowing us to rewrite:
		\begin{align*}
			\frac{d}{dt}( \mathbf{\dot{r}} \times \mathbf{\vec{L}} ) &= \mu \frac{d}{dt}\Big(\frac{\mathbf{\vec{r}}}{r}\Big)\\
			\text{Integrating:}\\
			\mathbf{\dot{r}} \times \mathbf{\vec{L}}  &= \mu \frac{\mathbf{\vec{r}}}{r}+ \mathbf{B}\\
			\text{Using dot product:}\\
			\mathbf{\vec{r}} \cdot \mathbf{\dot{r}} \times \mathbf{\vec{L}}  &= \mathbf{\vec{r}} \cdot \mu \frac{\mathbf{\vec{r}}}{r}+ \mathbf{\vec{r}} \cdot \mathbf{B}
		\end{align*}
		Since in general:
		\begin{align*}
			\vec{v} \cdot \vec{w} \times \vec{z} &= \vec{v} \times \vec{w} \cdot \vec{z}\\
			\text{and;}&\\
			\vec{b} \cdot \vec{b} &= b^2
		\end{align*}
		Thus:
		\begin{align*}
			\mathbf{\vec{r}} \times \mathbf{\dot{r}} \cdot \mathbf{\vec{L}}  &= \mu \frac{1}{r} \mathbf{\vec{r}} \cdot \mathbf{\vec{r}} + \mathbf{\vec{r}} \cdot \mathbf{B}\\
			\mathbf{\vec{L}} \cdot \mathbf{\vec{L}}  &= \mu \frac{1}{r} (r^2) + rB \cos \theta\\
			L^2 &= \frac{\mu}{r} + rB \cos \theta	
		\end{align*}
		Solving for r, we get the orbital equation;
		\begin{align*}
			r &= \frac{L^2}{\mu} \frac{1}{1 - (B/\mu) \cos \theta}
		\end{align*}
		where: 
		\begin{align*}
			\theta &= \text{the true anomaly} (\nu)
		\end{align*}
		In full form, the orbital equation is written as:
		\begin{align}
			r = \frac{L^2}{m^2\mu} \frac{1}{1 - e \cos \nu}
		\end{align}
		
		
		% v
		\pagebreak
		\item Draw the satellite orbit using latex\\\\\\
		\tikz \draw (0,0) ellipse (60mm and 120mm);
		
		
	\end{enumerate}
	
	
	\pagebreak
	\section*{Question 2}
	\paragraph*{Add a table in your Latex document with the following details}
	
	\begin{table}[ht]
		\centering
		\caption{SATELLITE NAME HAPA}
		{\renewcommand{\arraystretch}{2}
			\begin{tabular}{|m{0.01\linewidth}|m{0.35\linewidth}|m{0.55\linewidth}|} % {|c|c|c|} 
				%% BY NOW USHAJU AVENYE KUNAENDA NA HIZI TABLES
				\hline
				a. & Country of Operator/Owner & Githurai \\
				\hline
				b. & Users/Purpose & chiromo \\
				\hline
				c. & Class of Orbit &  chiromo \\
				\hline
				d. & Type of Orbit & chiromo \\
				\hline
				e. & Perigee (km) & 0.0 km \\
				\hline
				f. & Apogee (km) & 0.0 km \\
				\hline
				g. & Eccentricity & 0.0 \\
				\hline
				h. & Inclination ($^\circ$) &0.0$^\circ$ \\
				\hline
				i. & Period (minutes) & 0.0 minutes\\
				\hline
				j. & Launch Mass (kg) & 0 kg \\
				\hline
				k. & Date of Launch & 0$^{th}$ December, 0000\\
				\hline
				l. & Expected Lifetime (yrs.) & 0 yrs\\
				\hline
				m. & Launch Site & chiromo  \\
				\hline
				n. & Launch Vehicle &  	Mat za Rongai \\
				\hline
				o. & COSPAR Number & 0-0\\
				\hline
				p. & NORAD Number 4 & 0 \\
				\hline
		\end{tabular}}
	\end{table}
	
	
	
	\pagebreak
	\section*{Question 3}
	\paragraph*{Print your report including the Latex code from either Texstudio or overleaf and hand your report via google classroom}
	
	
	
\end{document}